Les lignes DC sont connectées par des prises \uD, à part au passage câbles bleus $\rightarrow$ Manganin.

On utilise les 17 lignes intérieures, c'est-à-dire pas les 4 coins de la prise.

\subsection{Soudure des prises uD}
Les prises \uD sont assez fragiles, il ne faut pas appuyer trop sur les pins avec le fer, au risque de les casser (rattrapable mais pas très pratique). La technique est de remplir la pin d'étain, puis de glisser le fil dedans sans avoir à apporter d'étain.

Il est préférable de mettre une gaine thermorétractable à une soudure sur deux (j’ai aussi mis de la grosse gaine thermo pour isoler les deux lignes).


\subsection{Connexion avec la canne}
On utilise une prise \uD (vis maison) que l'on fixe sur le bouchon de la canne. Faire attention au sens de branchement, en fonction de l'aménagement du cryostat (normalement un trait au feutre noir indique le sens).

\subsection{Tresse}
Entre le boîtier de thermalisation et de filtrage, les câbles sont blindés par une tresse d'aluminium. Il est préférable de faire passer les câbles une fois qu'une prise \uD est soudée.

\subsection{Thermalisation}
\subsubsection{Presse de thermalisation}
À l'étage 100mK, on thermalise les câbles de manganin à l'aide de la (double) presse dorée. On colle le tout à l'aide de Stycast

\begin{description}
    \item[Résine] Stycast (Emerson \& Cuming) 1kg
    \item[Durcisseur] Stycast (Emerson \& Cuming) 12g
\end{description}
Il faut utiliser $8\%$ de durcisseur dans le mélange.

\subsubsection{Boîtier de thermalisation}
Voilà.
