\documentclass[a4paper,11pt]{article}
\usepackage[T1]{fontenc}
\usepackage[utf8]{inputenc}
\usepackage{lmodern}
\usepackage[francais]{babel}

\title{Guide de câblage du cryostat à dillution}
\author{Félix Piédallu}
\date{Juin 2015}
\begin{document}

\maketitle
\tableofcontents

\begin{abstract}
%Dans le cadre de mon stage, j'ai été amené à continuer le câblage du cryostat à dillution de Laure Bruhat.
\end{abstract}

\section{Câblage des lignes DC}
\subsection{Bloc de filtrage}
\subsubsection{Connexions du bloc}
Le bloc est connecté grâce à des prises $\mu$D. Les vis d'entrée sont "maison", les vis de sortie sont des vis Allen 2.5mm.

\subsubsection{Compartimentage du bloc}
Afin de filtrer les micro-ondes des lignes DC, nous faisons passer les 17 câbles par un boîtier rempli d'Écosorb.

Malheureusement, l'Écosorb peut abîmer les soudures et les câbles au bout de quelques cycles de refroidissement. Nous avons donc décidé de compartimenter ce boîtier pour protéger les connexions.

Des pièces en PLA vont alors être imprimées. Elles ont été dessinées grâce à OpenSCAD et converties au format STL (que l'on peut trouver sur mon dépôt Git).
\end{document}
