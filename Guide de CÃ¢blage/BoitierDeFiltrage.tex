

\begin{BOM}
    \item 10 \fois Vis M2
    \item 2 \fois Prises \uD femelle
    \item 4 \fois Vis M1 + écrou + 2 rondelles (pour les prises)
    \item \fois 
\end{BOM}


\subsection{Connexions du bloc}
Le bloc est connecté grâce à des prises \uD. Les vis d'entrée sont "maison", les vis de sortie sont des vis Allen 2.5mm.

\subsection{Compartimentage du bloc}
Afin de filtrer les micro-ondes des lignes DC, nous faisons passer les 17 câbles par un boîtier rempli d'Écosorb.

Malheureusement, l'Écosorb peut abîmer les soudures et les câbles au bout de quelques cycles de refroidissement. Nous avons donc décidé de compartimenter ce boîtier pour protéger les connexions.

Des pièces en PLA vont alors être imprimées. Elles ont été dessinées grâce à OpenSCAD et converties au format STL. On peut trouver tout ça sur mon dépôt Git.

\subsection{Câblage du bloc}
On utilise 17 câbles bleus de 80cm. Ces câbles sont entortillés autour d'une chute de câble coaxial. On les passe alors d'abord dans les pièces en PLA puis on les soude sur les prises \uD.

\subsection{Préparation de l'Écosorb}
\begin{description}
    \item[Résine] Eccosorb (Emerson \& Cuming) CRS 117 PTA 1kg
    \item[Durcisseur] Eccosorb (Emerson \& Cuming) CRS PTB 12g
\end{description}
Il faut utiliser $1,18\%$ de durcisseur dans le mélange.

Ici on a mélangé 2,5g de catalyseur pour 212g de pâte. Il faut d'abord bien homogénéiser la résine (A) avant de mélanger au durcisseur (B).

Le mélange sèche en quelques jours. Il faut donc faire attention à poser le bloc bien à l'horizontale (en pensant aux prises).

