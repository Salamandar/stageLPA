Il faut enfin caractériser les câbles coaxiaux fabriqués au VNA afin :
\begin{itemize}
    \item de vérifier qu'ils n'ont pas été abîmés (mal cintrés)
    \item d'avoir les valeurs exactes d'atténuation des câbles à la fréquence de mesure, afin d'avoir une mesure la plus précise possible.
\end{itemize}

\subsection{Création de la nouvelle trace}
\begin{itemize}
    \item Il faut se placer dans une "fenêtre" libre (clic-droit > Créer fenêtre)
    \item Menu Trace > New Trace.
    \item Sélectionner les tracés correspondants aux ports utilisés (S33, S34, S43, S44 par exemple)
    \item Sélectionner un Channel disponible pour ne pas risquer d'influencer d'autres mesures sur d'autres fenêtres.
\end{itemize}

\subsection{Paramètres du VNA}
\begin{description}
    \item[Nombre de points :] 12801 (menu Sweep)
    \item[Puissance :] -20dB et Power On (Menu Power)
    \item[Gamme de mesure :] 1GHz - 20GHz et 4-8GHz
    \item[IF Bandwidth :] 1kHz (menu Avg)
\end{description}

\subsection{Calibration du VNA}
Avant toute mesure il faut calibrer le VNA. Nous utilisons la calibration électronique (Boîtier N4691-6006).

Menu Response > Cal Wizard

Use Electronic Calibration (ECal) > 2 Ports (sélectionner les ports branchés) > ECal Thru As... (do Orientation)

Cliquer sur Measure > Finish (pas Save).

\subsection{Lancement d'une mesure}
Vérifier que l'on est en Power On. Dans le menu Trigger, cliquer sur Single (mesure unique).

\subsection{Enregistrement d'une mesure}
File > Save As. Filetype : CSV.

Les fichiers sont nommés dans le format "yyyymmdd\_cableX\_GammeDeFréquences.csv".
