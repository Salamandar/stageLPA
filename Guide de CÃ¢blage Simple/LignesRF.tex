\subsection{Guide de fabrication des câbles}
Voici une liste des étapes à suivre pour fabriquer un câble coaxial connectorisé.

Il est préférable de cintrer le câble et de souder un connecteur avant de prendre les mesures et de couper le câble.

On utilisera du matériel des deux mallettes.

Il faut nettoyer le bout après chaque étape de limage/coupe avec de l'air sec.

\begin{description}
    \item[Cintrage du câble] On utilise la cintreuse. Pour chaque câble il faut faire un "U" pour éviter les interférences d'un étage à l'autre, et pour avoir une certaine souplesse du câble.
    
    Pour faire : 
    \begin{itemize}
        \item $1/4$ tour : il faut 15mm de câble
        \item $1/2$ tour : il faut 29mm
    \end{itemize}
    %TODO inclure un schéma
     \item[Dénudage] Il faut dénuder quelques millimètres du câble pour souder la pin sur l'âme du câble coaxial.
     On utilisera le support \textbf{21B} ainsi que la petite scie. Il faut aller doucement sans appuyer, jusqu'à ce qu'on sente que c'est "lisse".
     
     Ensuite, il faut retirer la gaine avec un scalpel et limer pour retirer les restes d'isolant et pour adoucir les angles.
     \item[Soudure de la pin centrale] On fixe la pin sur l'âme du câble, puis on serre le tout en place avec la pièce \textbf{W60} Il ne faut pas oublier l'entretoise \textbf{W56} entre la pin et l'isolant encore en place.
     
     Pour souder il suffit de chauffer l'extérieur de la pin tout en positionnant le fil d'étain sur le trou sur le bord de la pin.
     \item[Soudure de la prise extérieur] On fixe sur la prise mâle une prise femelle factice \textbf{W14M (81)} qui permet de positionner comme il faut la prise. Comme à l'étape précédente on serre le tout en place.
     
     Le plus efficace est de faire un tortillon d'étain au-dessus de la prise, que l'on chauffe. En étant un peu patient l'étain va fondre et rentrer naturellement dans la prise.
     \item[Fixation de l'isolant] La dernière étape est de mettre l'isolant entre la prise et la pin. on utilise la pièce \textbf{W52 (W53)}
     que l'on serre à la clé dynamométrique. On place l'isolant à l'intérieur, et on pousse d'un coup avec la pièce complémentaire.
     
     \item[Mesure du câble nécessaire] Maintenant il faut prendre la dimension de câble à couper.
     
     Sur le montage il faut prendre la dimension entre les deux 
     
\end{description}

\subsection{Thermalisation des câbles RF}

Les câbles RF se thermalisent grâce aux pinces dorées (sur les câbles \textit{et} sur les atténuateurs). On utilise l'Apiezon N pour avoir un bon contact thermique avec la pince.

Une des vis de chaque pince permet de fixer un fil de cuivre doré (elle est donc plus longue que les autres).

\begin{description}
    \item[Sur câble Rg405 :] 3 vis 10mm + 1 vis 16mm
    \item[Sur atténuateur :] 1 vis 16mm + 1 vis 2mm
\end{description}
